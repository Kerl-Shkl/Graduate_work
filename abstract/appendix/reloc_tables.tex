%! TEX root=../main.tex

%\chapter*{Приложение А}
\StdAppendixx{обязательное}
             {Таблицы с описанием структур из секции релокаций}
\label{app:tables}

\begin{table}[h!]
  \caption{Структура блока исправлений}
  \centering
  \label{tab:fixup_block}
  \begin{tabularx} {\textwidth} {
      | >{\raggedright \arraybackslash \hsize=.17\hsize}X 
      | >{\arraybackslash \hsize=.13\hsize}X
      | >{\arraybackslash \hsize=.25\hsize}X
      | >{\arraybackslash \hsize=.45\hsize}X|
    } 
    \hline 
    \multicolumn{1}{|c|}{Смещение} &
    \multicolumn{1}{c|}{Размер} &
    \multicolumn{1}{c|}{Поле} &
    \multicolumn{1}{c|}{Значение}  \\
    \hline 
    0 & 4 & Относительный виртуальный адрес страницы &
      Базовый адрес загрузки и относительный виртуальный адрес
      страницы прибавляется к каждому смещению в таблице, чтобы получить
      виртуальный адрес по которому необходимо провести исправление \\
    \hline
    4 & 4 & Размер блока исправлений &
      Общее количество байтов, занимаемых блоком исправлений, включая эту
      структуру.\\
    \hline
  \end{tabularx}  
\end{table}

\begin{xltabular}{\textwidth}{
      | >{\raggedright \arraybackslash \hsize=.17\hsize}X 
      | >{\raggedright \arraybackslash \hsize=.13\hsize}X
      | >{\arraybackslash \hsize=.25\hsize}X
      | >{\arraybackslash \hsize=.45\hsize}X|
  }

  \caption{Структура поля таблицы базовых релокаций}
  \label{tab:fixup_field}\\

  \hline
  \multicolumn{1}{|c|}{Смещение} &
  \multicolumn{1}{c|}{Размер} &
  \multicolumn{1}{c|}{Поле} &
  \multicolumn{1}{c|}{Значение}  \\
  \endfirsthead

  \multicolumn{4}{l}%
  {Окончание таблицы А.2} \\
  \hline
  \multicolumn{1}{|c|}{Смещение} &
  \multicolumn{1}{c|}{Размер} &
  \multicolumn{1}{c|}{Поле} &
  \multicolumn{1}{c|}{Значение}  \\
  \endhead
  
  \multicolumn{4}{r}{{Продолжение таблицы на следующей странице}} \\ 
  \endfoot

  \endlastfoot
  \hline
  0 & 4 бита & Тип & Значение, размещенное в старших четырех битах слова, 
  указывает на тип исправления. Всего типов исправлений может быть 16. Каждый
  тип указывает, как провести коррекцию значения в памяти.\\
  \hline
  0 & 12 бит & Смещение & Значение, размещенное в младших двенадцати битах
  слова, указывает смещение относительно относительного виртуального адреса
  страницы. Это смещение указывает место, где необходимо произвести
  коррекцию. \\ 
  \hline

\end{xltabular}

%\begin{table}[h!]
%  \centering
%  \caption{Структура поля таблицы базовых релокаций}
%  \begin{tabularx}{\textwidth}{
%      | >{\raggedright \arraybackslash \hsize=.17\hsize}X 
%      | >{\raggedright \arraybackslash \hsize=.13\hsize}X
%      | >{\arraybackslash \hsize=.25\hsize}X
%      | >{\arraybackslash \hsize=.45\hsize}X|
%    } 
%    \hline
%    \multicolumn{1}{|c|}{Смещение} &
%    \multicolumn{1}{c|}{Размер} &
%    \multicolumn{1}{c|}{Поле} &
%    \multicolumn{1}{c|}{Значение}  \\
%    \hline
%    0 & 4 бита & Тип & Значение, размещенное в старших четырех битах слова, 
%    указывает на тип исправления. Всего типов исправлений может быть 16. Каждый
%    тип указывает, как провести коррекцию значения в памяти.\\
%    \hline
%    0 & 12 бит & Смещение & Значение, размещенное в младших двенадцати битах
%    слова, указывает смещение относительно относительного виртуального адреса
%    страницы. Это смещение указывает место, где необходимо произвести
%    коррекцию. \\ 
%    \hline
%  \end{tabularx}
%  \label{tab:fixup_field}
%\end{table}
\thispagestyle{apppages_end}
