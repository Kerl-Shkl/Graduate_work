%! TEX root=../main.tex

%\chapter*{Приложение А}
\StdAppendix{обязательное}
            {Таблицы с описанием структур из секции релокаций}
\label{app:tables}

\begin{table}[h!]
  \centering
  \caption{Структура блока исправлений}
  \label{tab:fixup_block}
  \begin{tabularx} {\textwidth} {
      | >{\raggedright \arraybackslash \hsize=.17\hsize}X 
      | >{\arraybackslash \hsize=.13\hsize}X
      | >{\arraybackslash \hsize=.25\hsize}X
      | >{\arraybackslash \hsize=.45\hsize}X|
    } 
    \hline 
    \textbf{Смещение} & \textbf{Размер} & \textbf{Поле} & \textbf{Значение} \\
    \hline 
    0 & 4 & Относительный виртуальный адрес страницы &
      Базовый адрес загрузки и относительный виртуальный адрес
      страницы прибавляется к каждому смещению в таблице, чтобы получить
      виртуальный адрес по которому необходимо провести исправление \\
    \hline
    4 & 4 & Размер блока исправлений &
      Общее количество байтов, занимаемых блоком исправлений, включая эту
      структуру.\\
    \hline
  \end{tabularx}  
\end{table}

\begin{table}[h!]
  \centering
  \caption{Структура поля таблицы базовых релокаций}
  \begin{tabularx}{\textwidth}{
      | >{\raggedright \arraybackslash \hsize=.17\hsize}X 
      | >{\raggedright \arraybackslash \hsize=.13\hsize}X
      | >{\arraybackslash \hsize=.25\hsize}X
      | >{\arraybackslash \hsize=.45\hsize}X|
    } 
    \hline
    \textbf{Смещение} & \textbf{Размер} & \textbf{Поле} & \textbf{Значение} \\
    \hline
    0 & 4 бита & Тип & Значение, размещенное в старших четырех битах слова, 
    указывает на тип исправления. Всего типов исправлений может быть 16. Каждый
    тип указывает, как провести коррекцию значения в памяти.\\
    \hline
    0 & 12 бит & Смещение & Значение, размещенное в младших двенадцати битах
    слова, указывает смещение относительно относительного виртуального адреса
    страницы. Это смещение указывает место, где необходимо произвести
    коррекцию. \\ 
    \hline
  \end{tabularx}
  \label{tab:fixup_field}
\end{table}

