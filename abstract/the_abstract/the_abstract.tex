%! TEX root = ../main.tex
\chapter*{Реферат}

\begin{center}
  \ztotpages\ с.\,, \totalfigures\ рис.\,, \totaltables\ таб.\,, 
  \totalappendixCounters\ прил.\\
  \vspace{4mm}
  \MakeUppercase{отладчик, защита программ, WinAPI, антиотладочные техники,
  ассемблер, дизассемблер, базовый адрес загрузки, контрольная сумма,
  PE-формат.}
\end{center}

В данной работе проектируется механизм защиты программы от изменения ее
исходного кода. Механизм основан на нахождении контрольной суммы секции кода.
Алгоритм нахождения контрольной суммы реализован на языке ассемблера. Для
проведения тестирования написана программа на языке C. Реализованы модули на
языке C++, предоставляющие набор функций для работы с секциями PE-файла.

%\vspace{3cm}
%{\let\clearpage\relax \chapter*{bar}}

\chapter*{Abstract}
\begin{center}
  \ztotpages\ p.\,, \totalfigures\ pic.\,, \totaltables\ tab.\,, 
  \totalappendixCounters\ app.\\
  \vspace{4mm}
  \MakeUppercase{debugger, program protection, WinAPI, anti-debugging techniques,
  assembler, disassembler, base load address, cyclic redundancy check,
  PE-format.}
\end{center}

In this paper, a mechanism is designed to protect the program from changing the
source code. The mechanism is based on finding the CRC of text section. The
algorithm for finding the CRC is implemented in assembly language. A program for
testing is developed in C. Modules are implemented in C++ language, providing a
set of functions for working with sections of a PE-file.
