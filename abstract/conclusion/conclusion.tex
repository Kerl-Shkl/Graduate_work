%! TEX root=../main.tex

В результате выполнения работы разработан метод защиты программы от
изменения ее исходного кода.  Кроме того, предложенный метод одновременно
препятствует процессу отладки. Разработан и представлен алгоритм нахождения
контрольной суммы кода защищаемой секции. Составленный алгоритм реализован и
протестирован.  Результаты тестирования показали, что разработанный метод
обеспечивает защиту на достойном уровне.

Алгоритм реализован на языке ассемблера, так как была необходимость иметь
полный контроль над генерируемым машинным кодом. Реализация на языке ассемблера
позволила достичь независимости реализованного метода от компилятора.

Ассемблерный код был помещен в макрос языка C, что в свою очередь позволило
решить две проблемы. Во-первых, код защиты можно вставлять в тело программы
простым вызовом функции-макроса, но при этом в машинном коде будет не вызов
функции с передачей управления, а подстановка кода защиты. Во-вторых, можно
переназначать регистры процессора используемые при нахождении контрольной суммы.

Помимо этого, в ходе работы написаны программные модули на языке C++,
которые обеспечивают удобную работу с секциями PE-файла, нахождением контрольной
суммы для различных секций, а также замену заданной последовательности байт в
файле. 

Для тестирования полученного механизма написана демонстрационная программа на
языке C. В ходе тестирования механизм защиты смог обнаружить изменения в
исходном коде программы и оказать затруднение процессу отладки. При этом код
механизма защиты незаметен в теле остального кода программы.

