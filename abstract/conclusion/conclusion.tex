%! TEX root=../main.tex

В результате выполнения работы была разработан метод защиты программы от
изменения ее исходного кода, который также препятствует процессу отладки.  Был
разработан и представлен алгоритм нахождения контрольной суммы кода защищаемой
секции. Составленный алгоритм был реализован и протестирован.  Результаты
тестирования показали, что разработанный метод обеспечивает защиту на достойном
уровне.

Алгоритм был реализован на языке ассемблера, так как была необходимость иметь
полный контроль над получающимся машинным кодом. Реализация на языке ассемблера
позволила достичь независимости реализованного метода от компилятора.

Ассемблерный код был помещен в макрос языка C, что в свою очередь позволило
решить две проблемы. Во-первых, код защиты можно вставлять в тело программы
простым вызовом функции-макроса, но при этом в машинном коде будет не вызов
функции с передачей управления, а подстановка кода защиты. Во-вторых, можно
переназначать регистры процессора используемые при нахождении контрольной суммы.

Помимо этого в ходе работы были написаны программные модули на языке C++,
которые обеспечивают удобную работу с секциями PE-файла, нахождением контрольной
суммы для различных секций, а также замену заданной последовательности байт в
файле. 

Для тестирования полученного механизма была написана демонстрационная программа
на языке C. В ходе тестирования механизм защиты смог обнаружить изменения в
исходном коде программы, а также оказать затруднение процессу отладки. При этом
код механизма защиты обладает незаметностью относительно остального кода
программы.
