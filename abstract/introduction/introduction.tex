%! TEX root=../main.tex

В современном мире многие \textbf{TODO} \textit{общественные явления}
приобретают цифровую форму. Так, например: банковские операции, беспилотные
автомобили, умные дома, хранилища информации о частной жизни людей имеют
программно-информационные аспекты. В связи с этим становится очевидным
актуальность обеспечения информационной безопасности. 

Зачастую злоумышленники получают доступ к конфиденциальным данным по средствам
нахождения уязвимостей в программном обеспечении. Также злоумышленники могут
изменять исходный код программного обеспечения для устранения частей, отвечающих
за защиту, с целью дальнейшего незаконного распространения данной программы. И в
первом, и во втором случае применяется программа, называемая отладчиком. 

Изначально разработанные для упрощения процесса поиска ошибок в собственных
программах, отладчики получили широкое распространение как инструмент взлома. 

Целью данной работы является разработка методов обеспечения защиты программы от
отладчика. Данные методы должны как обеспечивать обнаружение факта работы
программы под отладчиком, так и препятствовать самому процессу отладки.

В связи с этим были поставлены следующие задачи:
\begin{itemize}
  \item Изучить внутреннее устройство программы отладчика.
  \item Разработать алгоритм защиты от отладчика.
  \item Реализовать полученный алгоритм, обеспечив при этом механизм достаточным
    уровнем скрытности.
  \item Провести тестирование полученной системы защиты.
\end{itemize}
