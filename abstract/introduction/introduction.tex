%! TEX root=../main.tex

В современном мире многие  общественные процессы приобретают цифровую форму.
Так, например, банковские операции, беспилотные автомобили, умные дома,
хранилища информации о частной жизни людей имеют программно-информационные
аспекты. В связи с этим становится очевидной актуальность обеспечения
информационной безопасности. 

Зачастую злоумышленники получают доступ к конфиденциальным данным по средствам
нахождения уязвимостей в программном обеспечении. Также злоумышленники могут
изменять исходный код программного обеспечения для устранения частей, отвечающих
за защиту, с целью дальнейшего незаконного распространения данной программы. И в
первом, и во втором случае применяется программа, называемая отладчиком. 

Изначально разработанные для упрощения процесса поиска ошибок в собственных
программах, отладчики получили широкое распространение как инструмент взлома. 

На российском рынке представлены две крупные организации, работа которых
посвящена проблеме информационной безопасности:
\begin{itemize}
  \item Лаборатория Касперского --- международная компания, работающая в сфере
    информационной безопасности и цифровой приватности.
  \item Hex-Rays, основанная Ильфаком Гульфановым, занимается разработкой
    инструментов двоичного анализа для рынка IT-безопасности.
\end{itemize}

Обе компании выпускают комплексные системы защиты, которые не интегрируются в
программу, а выступают сторонними модулями. В свою очередь, для обеспечения
защиты от конкретной угрозы со стороны отладчика было принято решение
разработать метод, который сможет как обеспечить обнаружение факта работы
программы под отладчиком, так и препятствовать самому процессу отладки.

%В свою очередь целью данной работы является разработка методов обеспечения
%защиты программы от взлома отладчиком. Данные методы должны как обеспечивать
%обнаружение факта работы программы под отладчиком, так и препятствовать самому
%процессу отладки.

В связи с этим были поставлены следующие задачи:
\begin{itemize}
  %\item Изучить внутреннее устройство программы отладчика.
  \item Изучить алгоритм действия отладчика.
  \item Разработать алгоритм защиты от отладчика.
  \item Реализовать полученный алгоритм, обеспечив при этом достаточный
    уровень скрытности.
  \item Провести тестирование полученной системы защиты.
\end{itemize}
