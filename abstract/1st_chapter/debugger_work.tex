\section{Принципы работы отладчика}
Отладчик~--- это программа, которая упрощает разработку программного
обеспечения, предоставляя разработчику способы поиска ошибок. Обычно функционал
отладчика предоставляет следующие возможнсти:
\begin{itemize}
  \item Поставить точку останова. Например, пометить интсрукцию, дойдя до
    которой, программа должна остановить свое выполнение и передать управление
    отладчику.
  \item Трассировать программу. То есть, последовательно выполнять инструкции, и
    после каждой останавливать выполнение программы и передавать управление
    отладчику.
  \item Отобразить состояние регисторов процессора на момент останова.
  \item Отобразить состояние стека процесса на момент останова.
\end{itemize}

Отладчик может самостоятельно запустить отлаживаемый процесс. В рассматриваемой
системе Microsoft Windows для этого нужно вызвать функцию \verb!CreateProcess!,
с указанием ей в качестве параметра \verb!fdwCreate! константы
\verb!DEBUG_PROCESS!. И также отладчик может подключиться к уже работающему
процессу. Для этого ему необходимо получить идентификатор процесса при помощи
функции \verb!OpenProcess!, после чего вызвать \verb!DebugActiveProcess! и таким
образом подключиться к нему.  Обычно отладчик открывает процесс с доступом на
чтение и запись в виртуальную память процесса.

Затем отладчик в цикле обрабатывает события отладки, используя функцию
\verb!WaitForDebugEvent!. После завершения обработки очередного события отладки
вызывает функцию \verb!ContinueDebugEvent!. Общую схему работу отладчика можно
представить следующим образом:
\begin{verbatim}
  CreateProcess("FileName.exe", ..., DEBUG_PROCESS, ...); 
  for (;;) {
    WaitForDebugEvent(&dbgEv, INFINITE);
    switch(dbgEv.dwDebugEventCode) 
    {
    case EXCEPTION_DEBUG_EVENT:
    ...
    }
    ContinueDebugEvent( dbgEv.dwProcessId,
                        dbgEv.dwThreadId,
                        dwContinueStatus );
  }
\end{verbatim}

Возможности предоставляемые отладчиком могут быть использованы злоумышленниками
для изучения уязвимостей программного обеспечения и обхода ограничений и защиты.
Для лучшего понимания способов защиты от отладчика рассмотрим, как работает
каждая из предоставляемых им возможностей более подробно.



