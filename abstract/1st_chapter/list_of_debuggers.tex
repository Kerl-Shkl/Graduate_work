%! TEX root = ../main.tex

\section{Наиболее распространенные отладчики}
Для лучшего противодействия отладчику полезно рассмотреть какие отладчики
используются на сегодняшний день. Исходя из этой информации можно организовать
простой способ защиты. Например, программа в процессе работы может просмотреть
процессы запущенные в системе, и, если среди них встретится какой-либо знакомый
отладчик, программа может изменить свое поведение или просто прервать
выполнение.

\subsection{SoftICE}
Наиболее важным свойством отладчика SoftICE является то, что он работает на
нулевом кольце защиты.

В архитектуе x86 есть три кольца защиты (ring-1, 2, 3).
Данные кольца предназначены для ограничения взаимодействия выполняющихся
программ между собой и с операционной системой. Как правило, на нулевом кольце
защиты выполняется сама операционная система, которая одна имеет доступ ко всем
привилегированным операциям. Рассматриваемый отладчик также располагается на
нулевом кольце защиты, что позволяет ему отлаживать не только пользовательские
приложения, но также драйвера и саму операционную систему.

Поддержка отладчика разработчиками прекратилась 11 июля 2007 года.

\subsection{IDA Pro}
IDA Pro --- интерактивный дизассемблер и универсальный отладчик. 

Дизассемблер --- программный инструмент, позволяющий получить из машинного кода
код на языке ассемблера. По принципу работы они делятся на пассивные и
интерактивные. Автоматические предоставляют пользователю готовый листинг
программы, а интерактивные позволяют на ходу изменять правила по которым
производится трансляция.

Как дизассемблер IDA Pro способен создавать карты выполнения фрагментов
программы, делая полученный код ассемблера еще более понятным человеком.В
качестве отладчика IDA Pro охватывает все рассмотренные ранее возможности
отладки, обеспечивает доступ ко всем сегментам пространства памяти процесса, а
также обеспечивает подробную визуализацию.

Проект активно поддерживается и развивается.

\subsection{WinDBG}
WinDBG --- отладчик предоставляемый фирмой Microsoft, предназначенный специально
для работы в операционной среде Windows. Является более мощной альтернативой
широко применяемому отадчику Visual Studio Debugger. Может использоваться как
отладчик режима. Имеет поддержку сторонних расширений. На данный момент является
одним из самых применяемых, благодаря своей универсальности.

\subsection{OllyDbg}
OllyDbg --- отладчик уровня приложений (ring-3). Помимо поддержки всех
рассматриваемых ранее возможностей отладки, больш\'ая часть мощности OllyDbg
заключается в расширениях, которые разрабатывают пользователи этого отладчика и
выкладывают в сеть Internet. OllyDbg имеет бесплатное распространение, а также
не требует установки. 

Для тестирования разработанных методов в данной работе применяется именно этот
отладчик.
