%! TEX = ../main.tex

\section{Отличия для 64-разрядных приложений}
Информация приведенная в предыдущих разделах справедлива для 32-битных
приложений. Чтобы реализовать выбранный метод в 64-битном приложении, нужно
произвести незначительные изменения.

PE-формат для 64-битных приложений называется PE32+. Отличия от классического PE
находятся в опциональном заголовке и таблице импорта. Чтобы алгоритм приведенный
в разделе \ref{sec:finding} корректно работал в 64-битных приложениях необходимо
изменить значения некоторых констант. 

В пункте \ref{item:dd_offset} массив с информацией о секциях расположен по
смещению в 112 байт. 
