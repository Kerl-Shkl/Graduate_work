%! TEX = ../main.tex
\section{Отличия для 64-разрядных приложений}
Информация приведенная в предыдущих разделах справедлива для 32-битных
приложений. Чтобы реализовать выбранный метод в 64-битном приложении, нужно
произвести незначительные изменения.

PE-формат для 64-битных приложений называется PE32+. Отличия от классического PE
находятся в опциональном заголовке и таблице импорта. Чтобы алгоритм приведенный
в разделе \ref{sec:finding} корректно работал в 64-битных приложениях необходимо
изменить одно значение. В пункте \ref{item:dd_offset} массив с информацией о
секциях расположен по смещению в 112 байт. 

Несколько меняется способ получения базового адреса загрузи программы. В
64-битной программе структура PEB расположена по смещению в \verb!0x60! байт
относительно регистра gs. А базовый адрес загрузки хранится по смещению в
\verb!0x10! байт внутри PEB. 

Таким образом, получить базовый адрес загрузки можно, используя следующий код:
\begin{verbatim}
      HMODULE hInst;
  #if defined _M_IX86
      __asm {
          mov eax, fs:0x30
          mov eax, [eax + 8]
          mov hInst, eax
      }
  #elif defined _M_X64
      BYTE* pPeb = NULL;
      pPeb = (BYTE*)__readgsqword(0x60);
      hInst = *(HMODULE*)(pPeb + 0x10);
  #endif
\end{verbatim}

В данном примере для получения адреса PEB применяется встроенная
функция компилятора MSVC \verb!__readgsqword!. Эта функция осуществляет чтение
памяти, указанного смещением относительно начала сегмента gs. Сделано это из-за
того, что MSVC не поддерживает ассемблерные вставки для 64-битных приложений.
