%! TEX root = ../main.tex

\section{Нахождение базового адреса загрузки}

Чтобы найти расположение нужной секции, нам также нужно знать базовый адрес
загрузки программы.

Самый очевидный способ получить базовый адрес загрузки --- это вызвать
соответствующую функцию Windows API. А именно, если вызвать функцию
\verb!GetModuleHandle! и в качестве параметра передать нулевой указатель, то
функция вернет переменную типа \verb!HMODULE!, значение которой будет
соответствовать базовому адресу загрузки программы.

Для нашей задачи такой подход имеет недостаток в виде системного вызова. Так как
отследить обращение к системным вызовам при помощи отладчика очень просто, такой
подход может поставить защиту под угрозу.

Чтобы получить базовый адрес загрузки программы без обращения к функциям API
Windows, обратимся к структуре PEB. PEB (process environment block) --- это
структура, которая содержит информацию о процессе, в памяти которого она
хранится. Процесс может получить доступ к этой структуре при помощи регистра fs
(для 64-битного процесса регистр gs). Получить адрес PEB можно, выполнив
следующий код:
\begin{verbatim}
  mov eax, fs:0x30
\end{verbatim}
После чего в регистре \verb!EAX! будет содержаться адрес PEB. По смещению в 8
байт в PEB хранится базовый адрес загрузки.

Таким образом можно получить базовый адрес загрузки, который необходим в
дальнейшем для определения границ секций, без обращения к системным вызовам.
