%! TEX root = ../main.tex

\section{Таблица базовых релокаций}

Как отмечалось ранее, операционная система Windows использует механизм ASLR.
ASLR (Address space layout randomization) --- это метод компьютерной
безопасности, предназначенный для предотвращения использования уязвимостей,
связанных с повреждением памяти. Данный прием может предотвратить простой
переход злоумышленника, например, к конкретной функции в памяти. ASLR случайным
образом упорядочивает позиции адресного пространства ключевых областей данных
процесса, включая базовый адрес загрузки, позиции стека, кучи и загружаемых
библиотек.

Из вышеизложенного следует, что адреса конкретных функций и переменных
невозможно определить до запуска программы. Следовательно, должен быть механизм
корректировки адресов в различных секциях программы. Например, необходимо
изменить все абсолютные адреса из инструкций \verb!jmp! в соответствии с
конкретным базовым адресом.

В системе Windows для этого в каждом исполняемом файле, который поддерживает
ASLR, есть таблица базовых релокаций. При помощи этой таблицы загрузчик
приложений Windows может скорректировать участки программы в соответствии с
адресом загрузки.

Таблица базовых релокаций содержит записи для всех исправлений в образе
программы. Общий размер таблицы содержится в опциональном заголовке. Вся таблица
разбита на блоки исправлений. Каждый блок содержит исправления для страницы
размера 4\,Кбайт и выровнен по 32-битной границе. 

Каждый блок исправлений начинается со структуры описанной в таблице 
A.\ref{tab:fixup_block}.

%\begin{table}[h!]
%  \centering
%  \caption{Структура блока исправлений}
%  \label{tab:fixup_block}
%  \begin{tabularx} {\textwidth} {
%      | >{\raggedright \arraybackslash \hsize=.17\hsize}X 
%      | >{\arraybackslash \hsize=.13\hsize}X
%      | >{\arraybackslash \hsize=.25\hsize}X
%      | >{\arraybackslash \hsize=.45\hsize}X|
%    } 
%    \hline 
%    \textbf{Смещение} & \textbf{Размер} & \textbf{Поле} & \textbf{Значение} \\
%    \hline 
%    0 & 4 & Относительный виртуальный адрес страницы &
%      Базовый адрес загрузки и относительный виртуальный адрес
%      страницы прибавляется к каждому смещению в таблице, чтобы получить
%      виртуальный адрес по которому необходимо провести исправление \\
%    \hline
%    4 & 4 & Размер блока исправлений &
%      Общее количество байтов, занимаемых блоком исправлений, включая эту
%      структуру.\\
%    \hline
%  \end{tabularx}  
%\end{table}

После описанной структуры следуют поля таблицы. Каждое поле занимает 2 байта и
имеет структуру, описанную в таблице A.\ref{tab:fixup_field}. 

%\begin{table}[h!]
%  \centering
%  \caption{Структура поля таблицы базовых релокаций}
%  \begin{tabularx}{\textwidth}{
%      | >{\raggedright \arraybackslash \hsize=.17\hsize}X 
%      | >{\raggedright \arraybackslash \hsize=.13\hsize}X
%      | >{\arraybackslash \hsize=.25\hsize}X
%      | >{\arraybackslash \hsize=.45\hsize}X|
%    } 
%    \hline
%    \textbf{Смещение} & \textbf{Размер} & \textbf{Поле} & \textbf{Значение} \\
%    \hline
%    0 & 4 бита & Тип & Значение, размещенное в старших четырех битах слова, 
%    указывает на тип исправления. Всего типов исправлений может быть 16. Каждый
%    тип указывает, как провести коррекцию значения в памяти.\\
%    \hline
%    0 & 12 бит & Смещение & Значение, размещенное в младших двенадцати битах
%    слова, указывает смещение относительно относительного виртуального адреса
%    страницы. Это смещение указывает место, где необходимо произвести
%    коррекцию. \\ 
%    \hline
%  \end{tabularx}
%  \label{tab:fixup_field}
%\end{table}

