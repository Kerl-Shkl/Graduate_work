%! TEX root = ../main.tex

\section{Нахождение необходимой секции в памяти}
\label{sec:finding}

Разрабатываемый метод будет вычислять контрольную сумму секции кода программы.
Путем несложных изменений можно адаптировать метод так, чтобы он искал
контрольную сумму секции данных, ресурсов или любой другой секции. Для этого
необходимо найти начало и размер конкретной секции в памяти. Чтобы решить эту
задачи можно воспользоваться знаниями об устройстве PE формата. 

Для нахождения нужной секции программа будет выполнять следующий алгоритм:
\begin{enumerate}[label=\arabic*.]
  \item Прибавить к базовому адресу загрузки 60 и прочитать четыре байта по этому
    адресу. По смещению в шестьдесят байт от базового адреса загрузки находится
    относительный виртуальный адрес PE-заголовка.
    
  \item Перейти к PE-заголовку, прибавив к базовому адресу прочитанные четыре
    байта. 

  \item Считать количество секций, которое содержится по смещению в 6 байт от
    PE-заголовка в двухбайтовом слове.

  \item По смещению в 20 байт от PE-заголовка находится двухбайтовое число, в
    котором содержится размер опционального заголовка. Это значение нам нужно
    для того, чтобы перейти к разделу секций.

  \item После этого необходимо перейти к опциональному заголовку, который
    находится по смещению в 24 байта от PE-заголовка. 

  \item \label{item:dd_offset} По смещению в 96 байт от опционального
    заголовка находится массив 8-ми байтовых значений. Первые четыре байта
    содержат адрес секции, а вторые четыре байта содержат размер секции. Пятый
    элемент этого массива соответствует таблице базовых релокаций, которая нам
    понадобится при подсчете контрольной суммы. Эти значения необходимо
    сохранить.

  \item После этого необходимо перейти к разделу секций, прибавив к началу
    опционального заголовка его размер, полученный на 4 этапе.

  \item \label{item:dd_struct} Раздел секций представляет собой массив структур,
    в которых находится информация о всех секциях, которые есть в программе.
    Необходимо циклом пройти по всем элементам данного массива, пока не
    встретится структура интересующей нас секции. Определить необходимую
    структуру можно по по полю характеристик секции. Так, например, секция кода
    содержит характеристику \verb!IMAGE_SCN_CNT_CODE! (\verb!0x00000020!).

  \item При нахождении интересующей секции необходимо сохранить ее относительный
    виртуальный адрес и размер.

\end{enumerate}

В итоге проделанных действий мы будем иметь относительные виртуальные адреса и
размеры искомой секции и таблицы базовых релокаций.


