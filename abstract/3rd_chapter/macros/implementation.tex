%! TEX root = ../macros.tex

\subsection{Реализация алгоритма}

В данной работе приведенный алгоритм реализован на языке ассемблера. Сам
ассемблерный код реализован в виде макроса. Преимуществом такого подхода
является то, что при вызове функции-макроса происходит не вызов функции, а
подстановка кода функции на место вызова. То есть это работает аналогично
встраиваемым функциям. Однако, если попытаться сделать функцию содержащую код
алгоритма, то компилятор проигнорирует ключевое слово \verb!inline!. Для
компилятора MSVC данная проблема решается ключевым словом \verb!__forceinline!,
но, например, если поменять компилятор на gcc, то такое решение уже не подойдет.
Для обеспечения универсальности кода по отношению к компилятору реализация
алгоритма помещена в макрос.

Использование макроса предоставляет еще одну возможность, а именно указать при
вызове макроса разный порядок регистров. Таким образом, вызывая макрос с разным
порядком регистров, будут получаться разные участки кода. Если злоумышленник
найдет и исправит один блок защиты, то найти остальные путем поиска
повторяющегося кода у него не получится.

