%! TEX root=../main.tex

\section{Принцип работы отладчика }

\paragraph{Слайд 3}\mbox{}\par

Рассмотрим принцип действия отладчика. Отладчик может самостоятельно запустить
отлаживаемый процесс или подключиться к уже работающему процессу. Как правило,
отладчик открывает процесс с доступом на чтение и запись в виртуальную память
процесса. Затем отладчик в цикле обрабатывает события отладки.

\paragraph{Слайд 4}\mbox{}\par
Отладчик предоставляет две основные возможности: поставить точку останова, дойдя
до которой выполнение программы прервется, или трассировать программу, то есть
выполнять ее по шагам. Чтобы поставить точку останова отладчик должен заменить
один байт в коде программы на инструкцию \verb!CC!. А чтобы продолжить
выполнение программы, ему необходимо восстановить замененный байт. 

Отладчик позволяет изменить исходный код программы во время ее выполнения, после
чего сохранить изменения в файл на диске. Таким образом получают взломанную
версию программы.

Исходя из ранее сказанного, было решено реализовать механизм защиты, который
будет препятствовать изменению исходного кода программы. Такой подход затруднит
использование точек останова при отладке, а также защитит исходный код программы
от модификации.
