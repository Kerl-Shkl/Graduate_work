%! TEX root = ../main.tex

\section{Итоговый алгоритм}

\paragraph{Слайд 8}\mbox{}\par

Алгоритм нахождения контрольной суммы должен быть быстрым, чтобы по задержке
нельзя было определить место в программе, где осуществляется проверка. Также
алгоритм не должен хранить в памяти большое количество промежуточных данных, так
как это тоже упрощает нахождение блока защиты.

Итоговый алгоритм состоит из главного цикла, в котором суммируется каждый байт
секции кода, кроме тех случаев, когда адрес проверяемого байта есть в таблице
релокаций. В этом случае этот байт и три следующих за ним игнорируются. 
Полученный алгоритм имеет линейную сложность, а также использует всего 16 байт
в стеке процесса.
