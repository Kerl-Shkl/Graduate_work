%! TEX root=../main.tex

\section{Вступление}

\paragraph{Слайд 2}\mbox{}\par
В связи с тем, что в современном мире многие общественные процессы приобретают
цифровую форму проблема информационной безопасности является крайне актуальной.
Наиболее мощным и распространенным Инструментом взлома является программа
отладчик.

\textbf{Целью данной работы} является разработка метода обеспечения защиты
программы от взлома отладчиком. Этот метод должен как обеспечивать обнаружение
факта работы программы под отладчиком, так и препятствовать самому процессу
отладки.

В связи с этим были поставлены следующие \textbf{задачи}:
\begin{enumerate}
  \item изучить алгоритм действия отладчика;
  \item разработать алгоритм защиты от отладчика;
  \item реализовать полученный алгоритм, обеспечив при этом достаточный уровень
    скрытности;
  \item провести тестирование полученной системы защиты.
\end{enumerate}

%Злоумышленники получают доступ к конфиденциальным данным по средствам нахождения
%уязвимостей в программном обеспечении. Также злоумышленники могут изменять
%исходный код программного обеспечения для устранения частей, отвечающих за
%защиту, с целью дальнейшего незаконного распространения данной программы. И в
%первом и во втором случае применяется программа, называемая отладчиком.

%Изначально разработанные для упрощения процесса поиска ошибок в собственных
%программах, отладчики получили широкое распространение как инструмент взлома.
