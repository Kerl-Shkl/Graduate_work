%! TEX root = ../main.tex

\section{Проблема ASLR (Слайды: 4, 5)}

\subsection{Содержание слайда}

\begin{itemize}
  \item Определение ASLR
  \item Инструкция near jmp
  \item Таблица базовых релокаций
\end{itemize}

\subsection{Текст}

Большинство современных операционных систем используют механизм ASLR. ASLR
(address space load randomization) --- это метод компьютерной безопасности,
предназначенный для предотвращения использования уязвимостей, связанных с
повреждением памяти. ASLR случайным образом упорядочивает позиции адресного
пространства ключевых областей данных процесса, включая базовый адрес загрузки,
позиции стека, кучи и загружаемых библиотек.

Из этого следует, что адреса конкретных функций и переменных невозможно
определить до запуска программы. Следовательно, необходим механизм
корректирования адресов в различных секциях программы. Например, необходимо
изменить все абсолютные адреса из инструкции \verb!near jmp! в соответствии с
конкретным базовым адресом.

В системе Windows для этого в исполняемом файле, есть таблица базовых
релокаций. Таблица базовых релокаций содержит записи для всех исправлений в
образе программы. 
