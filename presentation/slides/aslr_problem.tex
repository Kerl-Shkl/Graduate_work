%! TEX root = ../main.tex

\section{Проблема ASLR}

\paragraph{Слайд 6}\mbox{}\par

При реализации данного алгоритма пришлось столкнуться с проблемой вызванной
ASLR. ASLR (address space load randomization) --- это метод компьютерной
безопасности, предназначенный для предотвращения использования уязвимостей,
связанных с повреждением памяти. ASLR случайным образом упорядочивает позиции
адресного пространства ключевых областей данных процесса, включая базовый адрес
загрузки, позиции стека, кучи и загружаемых библиотек.

Из этого следует, что адреса конкретных функций и переменных невозможно
определить до запуска программы. А значит адреса указанные в параметрах
инструкций, как например \verb!near jmp!, в момент запуска программы меняются.
Следовательно, необходим механизм корректирования адресов в различных секциях
программы.

\paragraph{Слайд 7}\mbox{}\par

В системе Windows для этого в исполняемом файле, есть таблица базовых
релокаций. Таблица базовых релокаций содержит записи для всех исправлений в
образе программы. Защищаемая программа в процессе работы будет подсчитывать
контрольную сумму с учетом того, что некоторые адреса изменяются в момент
запуска программы.
