%!TEX root = ../main.tex

\section{Реализация алгоритма}

\paragraph{Слайд 9}\mbox{}\par

Составленный алгоритм реализован на языке ассемблера. Сам ассемблерный код
заключен в макрос. Преимуществом такого подхода является то, что при вызове
функции-макроса происходит не вызов функции, а подстановка кода функции на место
вызова. Использование макроса предоставляет еще одну полезную возможность, а
именно указать при вызове разный порядок регистров процессора. Таким образом,
вызывая макрос с разным порядком регистров, будут получаться разные участки
кода. Если злоумышленник найдет и исправит один блок защиты, то найти остальные
путем поиска повторяющихся элементов памяти у него не получится.


