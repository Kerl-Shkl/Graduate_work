%! TEX root = ../main.tex

\section{Классы для работы с секциями}

\paragraph{Слайд 10}\mbox{}\par

Помимо макроса для нахождения контрольной суммы в ходе работы были реализованы
модули, обеспечивающие нахождение контрольной суммы различных секций программ,
загруженных в память и расположенных на диске.

Данные модули следует использовать не для организации защиты, а для отладки. В
них реализованы:
\begin{itemize}
  \item алгоритмы нахождения контрольной суммы для любой секции программы;
  \item система логирования, настраиваемая при помощи директив препроцессора;
  \item возможность записи найденной контрольной суммы в исполняемый файл
    расположенный на диске.
\end{itemize}

На слайде представлена программа, которая подсчитывает контрольную сумму своей
секции кода. Причем подсчитывает сумму в своей виртуальной памяти, а также в exe
файле расположенном на диске.  Полученные значения совпадают.
