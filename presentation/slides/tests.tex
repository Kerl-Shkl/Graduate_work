%! TEX root = ../main.tex

\section{Тестирование}

\paragraph{Слайд 11}\mbox{}\par

Для проведения тестов была написана демонстрационная программа на языке C с
использованием только WinAPI. В случае, если введенный пользователем ключ не
совпадает с тем, который сгенерировала программа, выводится окно сообщения,
уведомляющее пользователя о неудаче.

\paragraph{Слайд 12}\mbox{}\par
При помощи отладчика OllyDBG был проведен взлом данной программы так, чтобы
любой введенный пароль считался корректным.

\paragraph{Слайд 13}\mbox{}\par
После этого в код программы был помещен разработанный механизм защиты. При
помощи отладчика были проведены аналогичные действия для взлома программы, но
программа распознала изменение в исходном коде осталась в заблокированном
состоянии.

\paragraph{Слайд 14}\mbox{}\par
Как видно на слайде, защищенная программа детектировала изменение в своем
исходном коде и осталась в заблокированном состоянии.

\paragraph{Слайд 15}\mbox{}\par
Помимо этого в ходе тестирования было подтверждено, что если установить в
отладчике точку останова программа детектирует изменение исходного кода и
изменит свое поведение, что сильно затруднит процесс отладки.
