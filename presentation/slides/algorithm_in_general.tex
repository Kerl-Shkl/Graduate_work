%! TEX root=../main.tex

\section{Алгоритм защиты от отладчика (Слайды: 3)}

\subsection{Содержание слайда}

\begin{itemize}
  \item PE-формат
  \item CRC
\end{itemize}

\subsection{Текст}

Разработанный алгоритм находит контрольную сумму критической секции программы,
после чего сравнивает ее с заранее известным значением. Если эти значения не
совпадают, значит код программы подвергался модификации. 

Для того, чтобы подсчитать контрольную сумму необходимо знать адреса начала и
конца критического участка. Метод реализованный в данной работе находит
контрольную сумму всей секции кода программы.

Информация о смещениях и размерах различных секций получается из знания
PE-формата. PE-формат --- формат исполняемых файлов, используемый в 32- и
64-разрядных версиях операционной системы Microsoft Windows. Формат PE
представляет собой структуру данных, содержащую всю информацию, необходимую
PE-загрузчику для отображения файла в память. Защищаемая программа во время
выполнения будет анализировать свой собственный PE-заголовок, чтобы получить
адрес начала и размер интересующей секции.
